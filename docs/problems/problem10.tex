\section{Problem 10: Minesweeper 1D (Codeforces 404 D- 1900 hard)}
\href{https://codeforces.com/contest/404/problem/D}{Link to problem}

\subsection{Postavka Problema}
Igra "Minesweeper 1D" se igra na liniji kvadrata, visine 1 kvadrat i širine $n$ kvadrata. Neki od kvadrata sadrže mine. Ako kvadrat ne sadrži minu, onda sadrži broj od 0 do 2 -- ukupan broj mina u susjednim kvadratima.

Na primjer, ispravno polje za igru izgleda ovako: 001*2***101*. Ćelije označene sa "*" sadrže mine. Primijetite da na ispravnom polju brojevi predstavljaju broj mina u susjednim ćelijama. Na primjer, polje 2* nije ispravno, jer ćelija sa vrijednošću 2 mora imati dvije susjedne ćelije sa minama.

Valera želi napraviti ispravno polje za igru "Minesweeper 1D". Već je nacrtao kvadratno polje širine $n$ ćelija, postavio nekoliko mina na polje i napisao brojeve u neke ćelije. Sada se pita na koliko načina može popuniti preostale ćelije minama i brojevima tako da na kraju dobije ispravno polje.

\subsubsection{Ulaz}
\begin{itemize}
    \item Prvi red sadrži niz karaktera bez razmaka $s_1s_2...s_n$ ($1 \leq n \leq 10^6$)
    \item Niz sadrži samo karaktere "*", "?" i cifre "0", "1" ili "2"
    \item Ako je karakter $s_i$ jednak "*", tada i-ta ćelija polja sadrži minu
    \item Ako je karakter $s_i$ jednak "?", tada Valera još nije odlučio šta staviti u i-tu ćeliju
    \item Karakter $s_i$ koji je jednak cifri predstavlja cifru napisanu u i-tom kvadratu
\end{itemize}

\subsubsection{Izlaz}
\begin{itemize}
    \item Ispisati jedan cijeli broj -- broj načina na koje Valera može popuniti prazne ćelije i dobiti ispravno polje
    \item Kako odgovor može biti velik, ispisati ga po modulu 1000000007 ($10^9 + 7$)
\end{itemize}

\subsection{Pristup Rješenju}
Problem se rješava koristeći dinamičko programiranje. Ključna opservacija je da za svaku ćeliju moramo pratiti broj mina u susjednim ćelijama kako bismo osigurali da brojevi u ćelijama budu ispravni.

\subsection{Dinamičko Programiranje}
Definišemo stanje dp[i][j][k] gdje je:
\begin{itemize}
    \item $i$ - pozicija do koje smo popunili polje
    \item $j$ - stanje prethodne ćelije (0-2 za broj, 3 za minu)
    \item $k$ - stanje trenutne ćelije (0-2 za broj, 3 za minu)
\end{itemize}

\subsection{Tranzicije Stanja}
Za svaku poziciju i stanje imamo sljedeće mogućnosti:
\begin{enumerate}
    \item Ako je trenutna ćelija "?", možemo staviti:
        \begin{itemize}
            \item Broj (0-2) ako je validan s obzirom na susjedne mine
            \item Minu (3) ako je validno s obzirom na susjedne brojeve
        \end{itemize}
    \item Ako je trenutna ćelija već popunjena, provjeravamo samo validnost
\end{enumerate}

\subsection{Analiza Kompleksnosti}
\begin{itemize}
    \item Vremenska Kompleksnost: $O(n \cdot 4 \cdot 4 \cdot 4)$
        \begin{itemize}
            \item Za svaku poziciju ($n$)
            \item Za svako stanje prethodne ćelije (4)
            \item Za svako stanje trenutne ćelije (4)
            \item Za svaku moguću vrijednost nove ćelije (4)
        \end{itemize}
    \item Prostorna Kompleksnost: $O(n \cdot 4 \cdot 4)$
        \begin{itemize}
            \item DP tabela: $O(n \cdot 4 \cdot 4)$
        \end{itemize}
\end{itemize}

\subsection{Detalji Implementacije}
Implementacija koristi nekoliko ključnih optimizacija:
\begin{itemize}
    \item Korištenje vrijednosti 3 za predstavljanje mine radi lakše obrade
    \item Efikasno računanje broja susjednih mina
    \item Provjera validnosti stanja prije ažuriranja DP tabele
    \item Pravilno rukovanje modularnom aritmetikom za velike brojeve
\end{itemize} 