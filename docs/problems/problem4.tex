\section{Problem 4: Lucky Common Subsequence (Codeforces 346B - 2000 hard)}
\href{https://codeforces.com/problemset/problem/346/B}{Link to problem}

\subsection{Postavka Problema}
Data su dva stringa $s_1$ i $s_2$ i treći string koji se zove virus. Zadatak je pronaći najdužu zajedničku podsekvenciju stringova $s_1$ i $s_2$ koja ne sadrži virus kao podstring.

Podsekvencija stringa je sekvenca koja se može dobiti iz originalnog stringa brisanjem nekih elemenata bez mijenjanja redoslijeda preostalih elemenata. Podstring je kontinuirana podsekvencija stringa.

\subsubsection{Ulaz}
\begin{itemize}
    \item Prvi red sadrži string $s_1$ ($1 \leq |s_1| \leq 100$)
    \item Drugi red sadrži string $s_2$ ($1 \leq |s_2| \leq 100$)
    \item Treći red sadrži string virus ($1 \leq |virus| \leq 100$)
    \item Svi stringovi se sastoje samo od velikih slova engleske abecede
\end{itemize}

\subsubsection{Izlaz}
\begin{itemize}
    \item Ispisati najdužu zajedničku podsekvenciju stringova $s_1$ i $s_2$ koja ne sadrži virus kao podstring
    \item Ako takva podsekvencija ne postoji, ispisati 0
\end{itemize}

\subsection{Pristup Rješenju}
Problem kombinuje dva klasična problema dinamičkog programiranja:
\begin{itemize}
    \item Najduža zajednička podsekvencija (LCS)
    \item Izbjegavanje podstringa (virus)
\end{itemize}

Ključni uvid je da moramo pratiti ne samo pozicije u $s_1$ i $s_2$, već i koliko smo virusa ``izgradili'' do sada.

\subsection{Dinamičko Programiranje}
Koristimo 3-dimenzionalni pristup dinamičkog programiranja:

\begin{equation*}
dp[i][j][k]
\end{equation*}

Gdje je:
\begin{itemize}
    \item $i$: Pozicija u prvom stringu $s_1$
    \item $j$: Pozicija u drugom stringu $s_2$
    \item $k$: Dužina prefiksa virusa koji se podudara sa sufiksom trenutne podsekvencije
\end{itemize}

\subsection{Tranzicije Stanja}
Za svaku poziciju $(i,j,k)$, imamo nekoliko mogućnosti:
\begin{enumerate}
    \item Ako se karakteri $s_1[i]$ i $s_2[j]$ podudaraju:
    \begin{itemize}
        \item Ako se karakter podudara sa sljedećim karakterom virusa, povećavamo $k$
        \item Inače, tražimo najduži prefiks virusa koji se podudara sa sufiksom + novi karakter
    \end{itemize}
    \item Preskočiti karakter iz $s_1$ ($i+1$)
    \item Preskočiti karakter iz $s_2$ ($j+1$)
    \item Preskočiti karaktere iz oba stringa ($i+1$, $j+1$)
\end{enumerate}

\subsection{Rekonstrukcija Rješenja}
Za rekonstrukciju rješenja koristimo dodatnu strukturu $previous\_state$ koja pamti prethodno stanje za svaku poziciju u DP tabeli. Kada pronađemo optimalno rješenje, pratimo put unazad da rekonstruišemo podsekvenciju.

\subsection{Analiza Kompleksnosti}
\begin{itemize}
    \item Vremenska Kompleksnost: $O(|s_1| \cdot |s_2| \cdot |virus|)$
    \item Prostorna Kompleksnost: $O(|s_1| \cdot |s_2| \cdot |virus|)$
\end{itemize}

\subsection{Detalji Implementacije}
Implementacija koristi nekoliko ključnih optimizacija:
\begin{itemize}
    \item Korištenje memset za brzu inicijalizaciju DP tabele
    \item Efikasno poređenje podstringova za virus pattern matching
    \item Rano odbacivanje nemogućih stanja
    \item Optimizovano praćenje prethodnih stanja za rekonstrukciju rješenja
\end{itemize} 