\section{Problem 9: Antimatter (Codeforces 383D - 2300 hard)}
\href{https://codeforces.com/problemset/problem/383/D}{Link to problem}

\subsection{Postavka Problema}
Iahub je slučajno otkrio tajni laboratorij. U njemu je pronašao $n$ uređaja poredanih u liniju, numerisanih od 1 do $n$ slijeva nadesno. Svaki uređaj $i$ ($1 \leq i \leq n$) može proizvesti ili $a_i$ jedinica materije ili $a_i$ jedinica antimaterije.

Iahub želi odabrati neki kontinuirani podniz uređaja u laboratoriji, odrediti način proizvodnje za svaki od njih (proizvodnja materije ili antimaterije) i na kraju napraviti fotografiju. Međutim, bit će uspješan samo ako su količine proizvedene materije i antimaterije u odabranom podnizu jednake (u suprotnom bi došlo do prekomjerne materije ili antimaterije na fotografiji).

\subsubsection{Ulaz}
\begin{itemize}
    \item Prvi red sadrži jedan cijeli broj $n$ ($1 \leq n \leq 1000$) -- broj uređaja
    \item Drugi red sadrži $n$ cijelih brojeva $a_1, a_2, ..., a_n$ ($1 \leq a_i \leq 1000$)
    \item Suma $a_1 + a_2 + ... + a_n$ će biti manja ili jednaka 10000
\end{itemize}

\subsubsection{Izlaz}
\begin{itemize}
    \item Ispisati jedan cijeli broj -- broj različitih načina na koje Iahub može napraviti fotografiju, po modulu $10^9 + 7$
\end{itemize}

\subsection{Pristup Rješenju}
Problem se rješava koristeći dinamičko programiranje. Ključna opservacija je da za svaki uređaj imamo dva izbora: proizvesti materiju ($+a_i$) ili antimateriju ($-a_i$). Cilj je pronaći sve kontinuirane podnizove gdje je suma odabranih vrijednosti jednaka nuli.

\subsection{Dinamičko Programiranje}
Definišemo stanje dp[i][s] gdje je:
\begin{itemize}
    \item $i$ - pozicija do koje smo obradili uređaje (0-bazirano)
    \item $s$ - trenutna razlika između količine materije i antimaterije
\end{itemize}

Koristimo tehniku "rolling array" za optimizaciju memorije, čuvajući samo dvije vrste stanja:
\begin{itemize}
    \item dp[current\_row][s] - stanja za trenutnu poziciju
    \item dp[next\_row][s] - stanja za sljedeću poziciju
\end{itemize}

\subsection{Tranzicije Stanja}
Za svaku poziciju $i$ i trenutnu sumu $s$, imamo sljedeće mogućnosti:
\begin{enumerate}
    \item Dodati materiju: dp[next\_row][s + a[i]] += dp[current\_row][s]
    \item Dodati antimateriju: dp[next\_row][s - a[i]] += dp[current\_row][s]
    \item Započeti novi podniz od trenutne pozicije: dp[next\_row][0] += 1
\end{enumerate}

\subsection{Analiza Kompleksnosti}
\begin{itemize}
    \item Vremenska Kompleksnost: $O(n \cdot MAXS)$
        \begin{itemize}
            \item Za svaku poziciju ($n$)
            \item Za svaku moguću sumu ($2 \cdot MAXS$)
            \item Gdje je MAXS = 10000
        \end{itemize}
    \item Prostorna Kompleksnost: $O(MAXS)$
        \begin{itemize}
            \item Koristimo samo dva reda DP tabele
            \item Svaki red ima veličinu $O(MAXS)$
        \end{itemize}
\end{itemize}

\subsection{Detalji Implementacije}
Implementacija koristi nekoliko ključnih optimizacija:
\begin{itemize}
    \item Korištenje "rolling array" tehnike za optimizaciju memorije
    \item Efikasno rukovanje negativnim indeksima pomoću pomaka u sredinu niza
    \item Pravilno rukovanje modularnom aritmetikom za velike brojeve
    \item Optimizovano čišćenje niza za sljedeću iteraciju
\end{itemize} 