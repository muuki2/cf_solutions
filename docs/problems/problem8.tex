\section{Problem 8: Bear and Company (Codeforces 790 C- 2500 hard)}
\href{https://codeforces.com/contest/790/problem/C/}{Link to problem}

\subsection{Postavka Problema}
Bear Limak priprema zadatke za programersko takmičenje. Naravno, bilo bi neprofesionalno spomenuti ime sponzora u tekstu zadatka. Limak to shvata ozbiljno i želi promijeniti neke riječi. Da bi tekst i dalje bio čitljiv, pokušat će modificirati svaku riječ što je manje moguće.

Limak ima string $s$ koji se sastoji od velikih slova engleske abecede. U jednom potezu može zamijeniti mjesta dvama susjednim slovima u stringu. Na primjer, može transformirati string "ABBC" u "BABC" ili "ABCB" u jednom potezu.

Limak želi dobiti string bez podstringa "VK" (tj. ne smije biti slova 'V' neposredno praćenog slovom 'K'). Može se lako dokazati da je to moguće za bilo koji početni string $s$.

\subsubsection{Ulaz}
\begin{itemize}
    \item Prvi red sadrži jedan cijeli broj $n$ ($1 \leq n \leq 75$) -- dužinu stringa
    \item Drugi red sadrži string $s$ koji se sastoji od velikih slova engleske abecede, dužine $n$
\end{itemize}

\subsubsection{Izlaz}
\begin{itemize}
    \item Ispisati jedan cijeli broj -- minimalan broj poteza koje Limak mora napraviti da bi dobio string bez podstringa "VK"
\end{itemize}

\subsection{Pristup Rješenju}
Problem se rješava koristeći dinamičko programiranje. Ključna opservacija je da možemo pratiti pozicije slova 'V', 'K' i ostalih slova odvojeno, te računati minimalan broj poteza potrebnih za njihovo preuređivanje.

\subsection{Dinamičko Programiranje}
Definišemo stanje dp[i][j][k][l] gdje je:
\begin{itemize}
    \item $i$ - broj obrađenih 'V' slova
    \item $j$ - broj obrađenih 'K' slova
    \item $k$ - broj obrađenih ostalih slova (označenih sa \texttt{\#})
    \item $l$ - flag koji označava da li je prethodno slovo bilo 'V' (1) ili nije (0)
\end{itemize}

\subsection{Tranzicije Stanja}
Za svako stanje imamo tri mogućnosti:
\begin{enumerate}
    \item Uzeti sljedeće 'V': dp[i+1][j][k][1]
    \item Uzeti sljedeće 'K' (samo ako prethodno nije bilo 'V'): dp[i][j+1][k][0]
    \item Uzeti sljedeće drugo slovo: dp[i][j][k+1][0]
\end{enumerate}

Cijena svake tranzicije se računa kao broj slova koja moramo preskočiti da bismo došli do željene pozicije.

\subsection{Analiza Kompleksnosti}
\begin{itemize}
    \item Vremenska Kompleksnost: $O(n^3)$
        \begin{itemize}
            \item Za svaku kombinaciju i, j, k (maksimalno n)
            \item Za svako stanje l (2 mogućnosti)
            \item Računanje cijene je $O(1)$ zbog binary searcha
        \end{itemize}
    \item Prostorna Kompleksnost: $O(n^3)$
        \begin{itemize}
            \item DP tabela: $O(n^3)$
            \item Pozicije slova: $O(n)$
        \end{itemize}
\end{itemize}

\subsection{Detalji Implementacije}
Implementacija koristi nekoliko ključnih optimizacija:
\begin{itemize}
    \item Pretprocesiranje stringa zamjenom svih slova osim 'V' i 'K' sa \texttt{\#}
    \item Korištenje binary searcha za efikasno računanje cijene tranzicija
    \item Pamćenje pozicija slova u sortiranim nizovima
    \item Korištenje upmin funkcije za ažuriranje minimalnih vrijednosti
\end{itemize} 