\documentclass[12pt,a4paper]{article}
\usepackage[utf8]{inputenc}
\usepackage[T1]{fontenc}
\usepackage[croatian]{babel}
\usepackage{lmodern}
\usepackage{amsmath}
\usepackage{amsfonts}
\usepackage{amssymb}
\usepackage{graphicx}
\usepackage{listings}
\usepackage{xcolor}
\usepackage{algorithm}
\usepackage{algpseudocode}
\usepackage{hyperref}

% Code listing style
\lstset{
    language=C++,
    basicstyle=\ttfamily\small,
    keywordstyle=\color{blue},
    stringstyle=\color{red},
    commentstyle=\color{green!60!black},
    numbers=left,
    numberstyle=\tiny,
    numbersep=5pt,
    frame=single,
    breaklines=true,
    breakatwhitespace=true,
    showstringspaces=false,
    inputencoding=utf8,
    extendedchars=true,
    literate={č}{{\v{c}}}1 {ć}{{\'{c}}}1 {đ}{{\dj}}1 {š}{{\v{s}}}1 {ž}{{\v{z}}}1
}

\title{Rješenja Codeforces Problema}
\author{Dokumentacija}
\date{2024-12-21}

\begin{document}

\maketitle
\tableofcontents

\section{Uvod}
Ovaj dokument sadrži detaljna rješenja i objašnjenja za 10 odabranih problema sa Codeforces platforme. Svaki problem je pažljivo analiziran i implementiran u C++ programskom jeziku. Za svako rješenje predstavljamo:

\begin{itemize}
    \item Detaljnu analizu problema i pristup rješenju
    \item Matematičku pozadinu i dokaze gdje je to potrebno
    \item Optimizovanu implementaciju sa jasnim objašnjenjima
    \item Analizu vremenske i prostorne složenosti
\end{itemize}

Svako rješenje je testirano na više test primjera kako bi se osigurala korektnost implementacije. 

\section{Problem 1: Cipher (Codeforces 156 C - 2000 hard)}
\href{https://codeforces.com/contest/156/problem/C}{Link to problem}

\subsection{Postavka Problema}
Sherlock Holmes je pronašao tajnu prepisku između dva VIP-a i odlučio je pročitati. Međutim, prepiska je šifrirana. Detektiv je pokušao dešifrirati prepisku, ali nije uspio razumjeti ništa.

Na kraju, nakon razmišljanja, smislio je nešto. Recimo da postoji riječ $s$, koja se sastoji od $|s|$ malih latiničnih slova. Tada za jednu operaciju možete odabrati određenu poziciju $p$ ($1 \leq p < |s|$) i izvršiti jednu od sljedećih radnji:

\begin{itemize}
    \item zamijeniti slovo $s_p$ s onim koje ga abecedno slijedi i zamijeniti slovo $s_{p+1}$ s onim koje ga abecedno prethodi;
    \item ili zamijeniti slovo $s_p$ s onim koje ga abecedno prethodi i zamijeniti slovo $s_{p+1}$ s onim koje ga abecedno slijedi.
\end{itemize}

Napominjemo da slovo "z" nema definirano sljedeće slovo, a slovo "a" nema definirano prethodno slovo. Zato odgovarajuće promjene nisu prihvatljive. Ako operacija zahtijeva izvođenje barem jedne neprihvatljive promjene, tada se takva operacija ne može izvesti.

Dve riječi se podudaraju u značenju ako se jedna od njih može transformirati u drugu kao rezultat nula ili više operacija.

Sherlock Holmes treba brzo naučiti odrediti sljedeće za svaku riječ: koliko riječi može postojati koje se podudaraju s njom u značenju, ali se razlikuju od nje u barem jednom znaku? Izračunajte ovaj broj za njega modulo 1000000007 ($10^9 + 7$).

\subsubsection{Ulaz}
\begin{itemize}
    \item Prvi red sadrži jedan cijeli broj $t$ ($1 \leq t \leq 10^4$) — broj testova.
    \item Sljedećih $t$ redova sadrži riječi, po jednu u svakom redu. Svaka riječ se sastoji od malih latiničnih slova i ima dužinu od 1 do 100, uključivo. Dužine riječi mogu se razlikovati.
\end{itemize}

\subsubsection{Izlaz}
\begin{itemize}
    \item Za svaku riječ ispišite broj različitih drugih riječi koje se podudaraju s njom u značenju — ne iz riječi navedenih u ulaznim podacima, već iz svih mogućih riječi. Kako traženi broj može biti vrlo velik, ispišite njegovu vrijednost modulo 1000000007 ($10^9 + 7$).
\end{itemize}

\subsection{Pristup Rješenju}
Problem se rješava koristeći dinamičko programiranje. Ključna opservacija je da možemo pratiti sumu pozicija slova u riječi i koristiti DP za izračunavanje broja mogućih transformacija.

\subsection{Dinamičko Programiranje}
Definišemo stanje dp[i][s] gdje je:
\begin{itemize}
    \item $i$ - dužina riječi
    \item $s$ - suma pozicija slova u riječi
\end{itemize}

\subsection{Tranzicije Stanja}
Za svaku poziciju $i$ i trenutnu sumu $s$, imamo sljedeće mogućnosti:
\begin{enumerate}
    \item Dodati slovo: dp[i+1][s + k] += dp[i][s] za svako slovo $k$
\end{enumerate}

\subsection{Analiza Kompleksnosti}
\begin{itemize}
    \item Vremenska Kompleksnost: $O(n \cdot 25n)$
        \begin{itemize}
            \item Za svaku dužinu riječi ($n$)
            \item Za svaku moguću sumu ($25n$)
        \end{itemize}
    \item Prostorna Kompleksnost: $O(n \cdot 25n)$
        \begin{itemize}
            \item DP tabela: $O(n \cdot 25n)$
        \end{itemize}
\end{itemize}

\subsection{Detalji Implementacije}
Implementacija koristi nekoliko ključnih optimizacija:
\begin{itemize}
    \item Korištenje modularne aritmetike za velike brojeve
    \item Efikasno računanje suma pozicija slova
    \item Pravilno rukovanje ulaznim i izlaznim podacima
\end{itemize}
 

\section{Problem 2: Clear The String (Codeforces 1132F - 2000 hard)}
\href{https://codeforces.com/problemset/problem/1132/F}{Link to problem}

\subsection{Postavka Problema}
Dat je string $s$ koji se sastoji od malih slova engleske abecede. U jednoj operaciji možemo izbrisati neki kontinualni podstring stringa $s$ ako su sva slova u tom podstringu jednaka. Na primjer, nakon brisanja podstringa bbbb iz stringa abbbaccda dobijamo string aaccda.

Zadatak je izračunati minimalan broj operacija potrebnih da se izbriše cijeli string $s$.

\subsubsection{Ulaz}
\begin{itemize}
    \item Prvi red sadrži jedan cijeli broj $n$ ($1 \leq n \leq 500$) - dužinu stringa $s$
    \item Drugi red sadrži string $s$ ($|s| = n$) koji se sastoji od malih slova engleske abecede
\end{itemize}

\subsubsection{Izlaz}
\begin{itemize}
    \item Ispisati jedan cijeli broj - minimalan broj operacija potrebnih da se izbriše string $s$
\end{itemize}

\subsection{Pristup Rješenju}
Problem rješavamo koristeći dinamičko programiranje. Ključni uvid je da za svaki podstring možemo:
\begin{itemize}
    \item Izbrisati prvo slovo zasebno
    \item Izbrisati prvo slovo zajedno sa nekim drugim istim slovom koje se nalazi kasnije u stringu
\end{itemize}

\subsection{Dinamičko Programiranje}
Definišemo stanje dp[l][r] kao minimalan broj operacija potrebnih da se izbriše podstring od pozicije l do pozicije r.

Imamo sljedeće slučajeve:
\begin{itemize}
    \item Ako je l > r: dp[l][r] = 0 (prazan string)
    \item Ako je l = r: dp[l][r] = 1 (jedno slovo)
    \item Inače, dp[l][r] je minimum od:
        \begin{itemize}
            \item 1 + dp[l+1][r] (brisanje prvog slova zasebno)
            \item dp[l+1][i-1] + dp[i][r] za svako i gdje je s[l] = s[i] (brisanje prvog slova sa nekim kasnijim istim slovom)
        \end{itemize}
\end{itemize}

\subsection{Tranzicije Stanja}
Za svaki podstring [l,r]:
\begin{enumerate}
    \item Prvo postavimo dp[l][r] = 1 + dp[l+1][r] (slučaj zasebnog brisanja)
    \item Zatim za svako i od l+1 do r:
        \begin{itemize}
            \item Ako je s[l] = s[i], pokušamo spojiti slovo na poziciji l sa slovom na poziciji i
            \item U tom slučaju moramo riješiti dva podproblema:
                \begin{itemize}
                    \item dp[l+1][i-1]: podstring između uparenih slova
                    \item dp[i][r]: podstring od drugog uparenog slova do kraja
                \end{itemize}
        \end{itemize}
\end{enumerate}

\subsection{Analiza Kompleksnosti}
\begin{itemize}
    \item Vremenska Kompleksnost: $O(n^3)$
        \begin{itemize}
            \item Imamo $O(n^2)$ stanja (za svaki par l,r)
            \item Za svako stanje prolazimo kroz $O(n)$ mogućih pozicija za uparivanje
        \end{itemize}
    \item Prostorna Kompleksnost: $O(n^2)$ za DP tabelu
\end{itemize}

\subsection{Detalji Implementacije}
Implementacija koristi nekoliko ključnih optimizacija:
\begin{itemize}
    \item Inicijalizacija DP tabele sa -1 za označavanje neposjećenih stanja
    \item Pravilno rukovanje baznim slučajevima (prazan string i pojedinačna slova)
    \item Efikasno računanje minimalnog broja operacija koristeći bottom-up pristup
\end{itemize} 

\section{Problem 3: Red-Green Towers (Codeforces 478D - 2000 hard)}
\href{https://codeforces.com/problemset/problem/478/D}{Link to problem}

\subsection{Postavka Problema}
\subsubsection{Opis}
Dat je broj $r$ crvenih i $g$ zelenih kocki. Potrebno je izgraditi toranj koji zadovoljava sljedeća pravila:
\begin{itemize}
    \item Toranj se sastoji od nekoliko nivoa
    \item Ako toranj ima $h$ nivoa, prvi nivo treba imati $h$ kocki, drugi nivo $h-1$ kocki, i tako dalje do vrha gdje zadnji nivo ima 1 kocku
    \item Svaki nivo mora biti izgrađen od kocki iste boje (ili sve crvene ili sve zelene)
\end{itemize}

\subsubsection{Ulaz}
\begin{itemize}
    \item Prvi i jedini red sadrži dva cijela broja $r$ i $g$ ($0 \leq r, g \leq 2 \cdot 10^5$, $r + g \geq 1$)
\end{itemize}

\subsubsection{Izlaz}
\begin{itemize}
    \item Ispisati jedan broj - broj različitih načina na koje se može izgraditi toranj maksimalne moguće visine, po modulu $10^9 + 7$
\end{itemize}

\subsection{Analiza Rješenja}
\subsubsection{Ključna Opažanja}
\begin{itemize}
    \item Za toranj visine $h$ potrebno je tačno $\frac{h(h+1)}{2}$ kocki
    \item Maksimalna visina tornja je najveće $h$ za koje vrijedi $\frac{h(h+1)}{2} \leq r + g$
    \item Dva tornja su različita ako postoji barem jedan nivo koji je u jednom tornju crven, a u drugom zelen
\end{itemize}

\subsubsection{Pristup Rješenju}
Problem rješavamo u dva koraka:
\begin{enumerate}
    \item Prvo određujemo maksimalnu moguću visinu tornja $h$
    \item Zatim koristimo dinamičko programiranje za računanje broja različitih načina izgradnje tornja te visine
\end{enumerate}

\subsection{Implementacija}
\subsubsection{Dinamičko Programiranje}
Definišemo stanje dp[t][r] kao broj načina da se izgradi prvih t nivoa tornja koristeći tačno r crvenih kocki.

Tranzicije su:
\begin{itemize}
    \item Dodavanje crvenog nivoa: dp[t+1][r + (t+1)] += dp[t][r]
    \item Dodavanje zelenog nivoa: dp[t+1][r] += dp[t][r]
\end{itemize}

\subsubsection{Optimizacije}
\begin{enumerate}
    \item \textbf{Memorijska Optimizacija}: Čuvamo samo dva reda DP tabele
    \item \textbf{Preskakanje Nula}: Preskačemo računanje tranzicija iz stanja gdje je dp[t][r] = 0
    \item \textbf{Modularna Aritmetika}: Sve operacije se izvode po modulu $10^9 + 7$
\end{enumerate}

\subsection{Analiza Složenosti}
\begin{itemize}
    \item \textbf{Vremenska složenost}: $O(h \cdot r)$
        \begin{itemize}
            \item Za svaki nivo t od 0 do h-1
            \item Za svaki broj iskorištenih crvenih kocki r
        \end{itemize}
    \item \textbf{Prostorna složenost}: $O(r)$
        \begin{itemize}
            \item Samo dva reda DP tabele
            \item Svaki red ima r+1 elemenata
        \end{itemize}
\end{itemize}


\section{Problem 4: Lucky Common Subsequence (Codeforces 346B - 2000 hard)}
\href{https://codeforces.com/problemset/problem/346/B}{Link to problem}

\subsection{Postavka Problema}
Data su dva stringa $s_1$ i $s_2$ i treći string koji se zove virus. Zadatak je pronaći najdužu zajedničku podsekvenciju stringova $s_1$ i $s_2$ koja ne sadrži virus kao podstring.

Podsekvencija stringa je sekvenca koja se može dobiti iz originalnog stringa brisanjem nekih elemenata bez mijenjanja redoslijeda preostalih elemenata. Podstring je kontinuirana podsekvencija stringa.

\subsubsection{Ulaz}
\begin{itemize}
    \item Prvi red sadrži string $s_1$ ($1 \leq |s_1| \leq 100$)
    \item Drugi red sadrži string $s_2$ ($1 \leq |s_2| \leq 100$)
    \item Treći red sadrži string virus ($1 \leq |virus| \leq 100$)
    \item Svi stringovi se sastoje samo od velikih slova engleske abecede
\end{itemize}

\subsubsection{Izlaz}
\begin{itemize}
    \item Ispisati najdužu zajedničku podsekvenciju stringova $s_1$ i $s_2$ koja ne sadrži virus kao podstring
    \item Ako takva podsekvencija ne postoji, ispisati 0
\end{itemize}

\subsection{Pristup Rješenju}
Problem kombinuje dva klasična problema dinamičkog programiranja:
\begin{itemize}
    \item Najduža zajednička podsekvencija (LCS)
    \item Izbjegavanje podstringa (virus)
\end{itemize}

Ključni uvid je da moramo pratiti ne samo pozicije u $s_1$ i $s_2$, već i koliko smo virusa ``izgradili'' do sada.

\subsection{Dinamičko Programiranje}
Koristimo 3-dimenzionalni pristup dinamičkog programiranja:

\begin{equation*}
dp[i][j][k]
\end{equation*}

Gdje je:
\begin{itemize}
    \item $i$: Pozicija u prvom stringu $s_1$
    \item $j$: Pozicija u drugom stringu $s_2$
    \item $k$: Dužina prefiksa virusa koji se podudara sa sufiksom trenutne podsekvencije
\end{itemize}

\subsection{Tranzicije Stanja}
Za svaku poziciju $(i,j,k)$, imamo nekoliko mogućnosti:
\begin{enumerate}
    \item Ako se karakteri $s_1[i]$ i $s_2[j]$ podudaraju:
    \begin{itemize}
        \item Ako se karakter podudara sa sljedećim karakterom virusa, povećavamo $k$
        \item Inače, tražimo najduži prefiks virusa koji se podudara sa sufiksom + novi karakter
    \end{itemize}
    \item Preskočiti karakter iz $s_1$ ($i+1$)
    \item Preskočiti karakter iz $s_2$ ($j+1$)
    \item Preskočiti karaktere iz oba stringa ($i+1$, $j+1$)
\end{enumerate}

\subsection{Rekonstrukcija Rješenja}
Za rekonstrukciju rješenja koristimo dodatnu strukturu $previous\_state$ koja pamti prethodno stanje za svaku poziciju u DP tabeli. Kada pronađemo optimalno rješenje, pratimo put unazad da rekonstruišemo podsekvenciju.

\subsection{Analiza Kompleksnosti}
\begin{itemize}
    \item Vremenska Kompleksnost: $O(|s_1| \cdot |s_2| \cdot |virus|)$
    \item Prostorna Kompleksnost: $O(|s_1| \cdot |s_2| \cdot |virus|)$
\end{itemize}

\subsection{Detalji Implementacije}
Implementacija koristi nekoliko ključnih optimizacija:
\begin{itemize}
    \item Korištenje memset za brzu inicijalizaciju DP tabele
    \item Efikasno poređenje podstringova za virus pattern matching
    \item Rano odbacivanje nemogućih stanja
    \item Optimizovano praćenje prethodnih stanja za rekonstrukciju rješenja
\end{itemize} 

\section{Problem 5: Cards and Joy (Codeforces 999F - 2000 hard)}
\href{https://codeforces.com/problemset/problem/999/F}{Link to problem}

\subsection{Postavka Problema}
Dato je $n$ igrača koji sjede za stolom. Svaki igrač ima svoj omiljeni broj. Omiljeni broj $j$-tog igrača je $f_j$.

Na stolu se nalazi $k \cdot n$ karata. Svaka karta sadrži jedan cijeli broj: $i$-ta karta sadrži broj $c_i$. Također, data je sekvenca $h_1, h_2, \ldots, h_k$ čije će značenje biti objašnjeno u nastavku.

Igrači moraju raspodijeliti sve karte tako da svaki od njih dobije tačno $k$ karata. Nakon što su sve karte raspodijeljene, svaki igrač prebroji koliko karata ima sa svojim omiljenim brojem. Nivo sreće igrača je $h_t$ ako igrač ima $t$ karata sa svojim omiljenim brojem. Ako igrač nema nijednu kartu sa svojim omiljenim brojem (tj. $t=0$), njegov nivo sreće je 0.

\subsubsection{Ulaz}
\begin{itemize}
    \item Prvi red sadrži dva cijela broja $n$ i $k$ ($1 \leq n \leq 500, 1 \leq k \leq 10$)
    \item Drugi red sadrži $k \cdot n$ cijelih brojeva $c_1, c_2, \ldots, c_{k \cdot n}$ ($1 \leq c_i \leq 10^5$)
    \item Treći red sadrži $n$ cijelih brojeva $f_1, f_2, \ldots, f_n$ ($1 \leq f_j \leq 10^5$)
    \item Četvrti red sadrži $k$ cijelih brojeva $h_1, h_2, \ldots, h_k$ ($1 \leq h_t \leq 10^5$)
\end{itemize}

\subsubsection{Izlaz}
\begin{itemize}
    \item Ispisati jedan cijeli broj -- maksimalan mogući ukupni nivo sreće igrača nakon raspodjele karata
\end{itemize}

\subsection{Pristup Rješenju}
Problem se može riješiti korištenjem dinamičkog programiranja. Ključni uvid je da možemo nezavisno rješavati za svaki omiljeni broj koji se pojavljuje među igračima.

\subsection{Dinamičko Programiranje}
Za svaki jedinstveni omiljeni broj $x$, rješavamo podproblem:
\begin{itemize}
    \item Imamo $p$ igrača koji žele broj $x$
    \item Imamo $c$ karata sa brojem $x$
    \item Svaki igrač mora dobiti između 0 i $k$ karata
\end{itemize}

Definiramo stanje dinamičkog programiranja:
\begin{equation*}
dp[i][j]
\end{equation*}

Gdje je:
\begin{itemize}
    \item $i$: Broj igrača kojima smo dodijelili karte (0 do $p$)
    \item $j$: Broj karata koje smo iskoristili (0 do $\min(c, p \cdot k)$)
\end{itemize}

\subsection{Tranzicije Stanja}
Za svako stanje $(i,j)$, razmatramo sve moguće načine da dodijelimo karte sljedećem igraču:
\begin{equation*}
dp[i+1][j+t] = \max(dp[i+1][j+t], dp[i][j] + h_t)
\end{equation*}

Gdje je:
\begin{itemize}
    \item $t$: Broj karata koje dajemo sljedećem igraču (0 do $\min(k, c-j)$)
    \item $h_t$: Nivo sreće za dobijanje $t$ karata
\end{itemize}

\subsection{Rekonstrukcija Rješenja}
Za svaki omiljeni broj $x$:
\begin{enumerate}
    \item Izračunamo optimalno rješenje za igrače koji žele taj broj
    \item Dodamo maksimalnu moguću sreću na ukupni rezultat
\end{enumerate}

\subsection{Analiza Kompleksnosti}
\begin{itemize}
    \item Vremenska Kompleksnost: $O(C \cdot N \cdot K \cdot N)$, gdje je $C$ broj različitih omiljenih brojeva
    \item Prostorna Kompleksnost: $O(N \cdot NK)$
\end{itemize}

\subsection{Detalji Implementacije}
Implementacija koristi nekoliko optimizacija:
\begin{itemize}
    \item Brojanje karata i igrača umjesto praćenja pojedinačnih dodjela
    \item Ograničavanje broja dostupnih karata na $\min(card\_count[x], players \cdot k)$
    \item Korištenje lokalnog DP-a za svaki omiljeni broj
    \item Rano odbacivanje nemogućih stanja
\end{itemize} 

\section{Problem 6: Supermarket Shopping (Codeforces 815C - 2400 hard)}
\href{https://codeforces.com/problemset/problem/815/C}{Link to problem}

\subsection{Postavka Problema}
Karen je studentica koja želi kupiti namirnice u supermarketu sa ograničenim budžetom. Supermarket nudi n proizvoda i posebne kupone za svaki proizvod. Problem je optimizirati kupovinu kako bi se kupio maksimalan broj proizvoda uz data ograničenja.

\subsubsection{Ulaz}
\begin{itemize}
    \item Prvi red sadrži dva cijela broja $n$ i $b$ ($1 \leq n \leq 5000$, $1 \leq b \leq 10^9$)
        \begin{itemize}
            \item $n$ - broj proizvoda u supermarketu
            \item $b$ - Karenin budžet
        \end{itemize}
    \item Sljedećih $n$ redova opisuju proizvode. Za svaki proizvod $i$:
        \begin{itemize}
            \item Dva cijela broja $c_i$ i $d_i$ ($1 \leq d_i < c_i \leq 10^9$)
                \begin{itemize}
                    \item $c_i$ - cijena proizvoda
                    \item $d_i$ - iznos popusta sa kuponom
                \end{itemize}
            \item Za $i \geq 2$, dodatni broj $x_i$ ($1 \leq x_i < i$) - indeks kupona koji mora biti iskorišten prije kupona $i$
        \end{itemize}
\end{itemize}

\subsubsection{Izlaz}
\begin{itemize}
    \item Jedan cijeli broj - maksimalan broj proizvoda koji Karen može kupiti sa budžetom $b$
\end{itemize}

\subsection{Pristup Rješenju}
Problem se može riješiti korištenjem dinamičkog programiranja i DFS pristupa. Ključni uvidi su:
\begin{itemize}
    \item Za svaki proizvod imamo dva stanja: kupovina sa kuponom i bez kupona
    \item Kuponi formiraju usmjereni aciklički graf (DAG) zavisnosti
    \item Za korištenje kupona $i$, moramo koristiti kupon $x_i$
\end{itemize}

\subsection{Dinamičko Programiranje}
Koristimo tri stanja za svaki čvor u grafu:
\begin{equation*}
dp[state][node]
\end{equation*}

Gdje je:
\begin{itemize}
    \item $state$ može biti:
        \begin{itemize}
            \item 0: bez korištenja kupona za trenutni čvor
            \item 1: sa korištenjem kupona za trenutni čvor
            \item 2: najbolji rezultat kombinujući oba stanja
        \end{itemize}
    \item $node$: trenutni proizvod koji razmatramo
\end{itemize}

\subsection{DFS i Spajanje Rezultata}
Za svaki čvor u grafu:
\begin{enumerate}
    \item Inicijaliziramo bazne slučajeve:
        \begin{itemize}
            \item Bez kupona: $\{0, cijena\}$
            \item Sa kuponom: $\{INF, cijena - popust\}$
        \end{itemize}
    \item Rekurzivno obrađujemo sve zavisne čvorove
    \item Spajamo rezultate koristeći funkciju merge:
        \begin{itemize}
            \item Za stanje bez kupona: merge(dp[0][node], dp[0][child])
            \item Za stanje sa kuponom: merge(dp[1][node], dp[2][child])
        \end{itemize}
\end{enumerate}

\subsection{Analiza Složenosti}
\begin{itemize}
    \item \textbf{Vremenska složenost}: $O(n \cdot m)$
        \begin{itemize}
            \item $n$ - broj proizvoda
            \item $m$ - maksimalan broj mogućih kombinacija proizvoda
        \end{itemize}
    \item \textbf{Prostorna složenost}: $O(n)$
        \begin{itemize}
            \item Prostor za graf zavisnosti
            \item DP tabela za svaki čvor
        \end{itemize}
\end{itemize}

\subsection{Detalji Implementacije}
\begin{itemize}
    \item Efikasno spajanje rezultata koristeći min operaciju
    \item Pamćenje samo neophodnih međurezultata
    \item Korištenje konstante \texttt{0x3f3f3f3f} kao beskonačnosti umjesto \texttt{INT\_MAX}
    \item Rano odbacivanje nemogućih stanja
\end{itemize} 

\section{Problem 7: Divide by Three (Codeforces 792C - 2000 hard)}
\href{https://codeforces.com/problemset/problem/792/C}{Link to problem}

\subsection{Postavka Problema}
Na tabli je napisan pozitivan cijeli broj n. Potrebno je transformisati taj broj u lijep broj brisanjem nekih cifara, pri čemu želimo obrisati što je moguće manje cifara.

Broj se smatra lijepim ako se sastoji od najmanje jedne cifre, nema vodećih nula i djeljiv je sa 3. Na primjer, 0, 99, 10110 su lijepi brojevi, dok 00, 03, 122 nisu.

Potrebno je napisati program koji će za dati broj n pronaći lijep broj takav da se n može transformisati u taj broj brisanjem najmanjeg mogućeg broja cifara. Možete obrisati proizvoljni skup cifara, ne moraju biti uzastopne u broju n.

Ako nije moguće dobiti lijep broj, ispisati -1. Ako postoji više rješenja, ispisati bilo koje od njih.

\subsubsection{Ulaz}
\begin{itemize}
    \item Prvi red sadrži pozitivan cijeli broj n bez vodećih nula ($1 \leq n < 10^{100000}$)
\end{itemize}

\subsubsection{Izlaz}
\begin{itemize}
    \item Ispisati jedan broj -- bilo koji lijep broj dobijen brisanjem najmanjeg mogućeg broja cifara
    \item Ako ne postoji rješenje, ispisati -1
\end{itemize}

\subsection{Pristup Rješenju}
Problem se rješava koristeći dinamičko programiranje. Ključna opservacija je da možemo pratiti ostatak pri dijeljenju sa 3 za svaki prefiks i dužinu formiranog broja. Također, moramo voditi računa o vodećim nulama i posebnim slučajevima.

\subsection{Dinamičko Programiranje}
Definišemo stanje dp[i][r] kao maksimalnu dužinu lijepog broja koji se može formirati koristeći prvih i cifara sa ostatkom r pri dijeljenju sa 3. Dodatno, koristimo niz choice[i][r] za rekonstrukciju rješenja.

\subsection{Tranzicije Stanja}
Za svaku poziciju i i trenutnu cifru d, imamo sljedeće mogućnosti:
\begin{enumerate}
    \item Ako d nije 0, možemo početi novi broj: dp[i+1][d mod 3] = 1
    \item Za svaki postojeći ostatak r:
        \begin{itemize}
            \item Dodaj cifru: dp[i+1][(r · 10 + d) mod 3] = dp[i][r] + 1
            \item Ne dodaj cifru: dp[i+1][r] = dp[i][r]
        \end{itemize}
\end{enumerate}

\subsection{Analiza Kompleksnosti}
\begin{itemize}
    \item Vremenska Kompleksnost: O(n), gdje je n dužina ulaznog stringa
        \begin{itemize}
            \item Za svaku poziciju razmatramo 3 moguća ostatka
            \item Svaka operacija je O(1)
        \end{itemize}
    \item Prostorna Kompleksnost: O(n)
        \begin{itemize}
            \item dp tabela: O(n)
            \item choice niz: O(n)
        \end{itemize}
\end{itemize}

\subsection{Detalji Implementacije}
Implementacija koristi nekoliko ključnih optimizacija:
\begin{itemize}
    \item Korištenje pomoćne funkcije updateMax za efikasno ažuriranje dp stanja
    \item Pažljivo rukovanje vodećim nulama u izlazu
    \item Pravilno računanje ostataka tokom rekonstrukcije
    \item Optimizovano praćenje prethodnih stanja za rekonstrukciju rješenja
\end{itemize}
 

\section{Problem 8: Bear and Company (Codeforces 790 C- 2500 hard)}
\href{https://codeforces.com/contest/790/problem/C/}{Link to problem}

\subsection{Postavka Problema}
Bear Limak priprema zadatke za programersko takmičenje. Naravno, bilo bi neprofesionalno spomenuti ime sponzora u tekstu zadatka. Limak to shvata ozbiljno i želi promijeniti neke riječi. Da bi tekst i dalje bio čitljiv, pokušat će modificirati svaku riječ što je manje moguće.

Limak ima string $s$ koji se sastoji od velikih slova engleske abecede. U jednom potezu može zamijeniti mjesta dvama susjednim slovima u stringu. Na primjer, može transformirati string "ABBC" u "BABC" ili "ABCB" u jednom potezu.

Limak želi dobiti string bez podstringa "VK" (tj. ne smije biti slova 'V' neposredno praćenog slovom 'K'). Može se lako dokazati da je to moguće za bilo koji početni string $s$.

\subsubsection{Ulaz}
\begin{itemize}
    \item Prvi red sadrži jedan cijeli broj $n$ ($1 \leq n \leq 75$) -- dužinu stringa
    \item Drugi red sadrži string $s$ koji se sastoji od velikih slova engleske abecede, dužine $n$
\end{itemize}

\subsubsection{Izlaz}
\begin{itemize}
    \item Ispisati jedan cijeli broj -- minimalan broj poteza koje Limak mora napraviti da bi dobio string bez podstringa "VK"
\end{itemize}

\subsection{Pristup Rješenju}
Problem se rješava koristeći dinamičko programiranje. Ključna opservacija je da možemo pratiti pozicije slova 'V', 'K' i ostalih slova odvojeno, te računati minimalan broj poteza potrebnih za njihovo preuređivanje.

\subsection{Dinamičko Programiranje}
Definišemo stanje dp[i][j][k][l] gdje je:
\begin{itemize}
    \item $i$ - broj obrađenih 'V' slova
    \item $j$ - broj obrađenih 'K' slova
    \item $k$ - broj obrađenih ostalih slova (označenih sa \texttt{\#})
    \item $l$ - flag koji označava da li je prethodno slovo bilo 'V' (1) ili nije (0)
\end{itemize}

\subsection{Tranzicije Stanja}
Za svako stanje imamo tri mogućnosti:
\begin{enumerate}
    \item Uzeti sljedeće 'V': dp[i+1][j][k][1]
    \item Uzeti sljedeće 'K' (samo ako prethodno nije bilo 'V'): dp[i][j+1][k][0]
    \item Uzeti sljedeće drugo slovo: dp[i][j][k+1][0]
\end{enumerate}

Cijena svake tranzicije se računa kao broj slova koja moramo preskočiti da bismo došli do željene pozicije.

\subsection{Analiza Kompleksnosti}
\begin{itemize}
    \item Vremenska Kompleksnost: $O(n^3)$
        \begin{itemize}
            \item Za svaku kombinaciju i, j, k (maksimalno n)
            \item Za svako stanje l (2 mogućnosti)
            \item Računanje cijene je $O(1)$ zbog binary searcha
        \end{itemize}
    \item Prostorna Kompleksnost: $O(n^3)$
        \begin{itemize}
            \item DP tabela: $O(n^3)$
            \item Pozicije slova: $O(n)$
        \end{itemize}
\end{itemize}

\subsection{Detalji Implementacije}
Implementacija koristi nekoliko ključnih optimizacija:
\begin{itemize}
    \item Pretprocesiranje stringa zamjenom svih slova osim 'V' i 'K' sa \texttt{\#}
    \item Korištenje binary searcha za efikasno računanje cijene tranzicija
    \item Pamćenje pozicija slova u sortiranim nizovima
    \item Korištenje upmin funkcije za ažuriranje minimalnih vrijednosti
\end{itemize} 

\section{Problem 9: Antimatter (Codeforces 383D - 2300 hard)}
\href{https://codeforces.com/problemset/problem/383/D}{Link to problem}

\subsection{Postavka Problema}
Iahub je slučajno otkrio tajni laboratorij. U njemu je pronašao $n$ uređaja poredanih u liniju, numerisanih od 1 do $n$ slijeva nadesno. Svaki uređaj $i$ ($1 \leq i \leq n$) može proizvesti ili $a_i$ jedinica materije ili $a_i$ jedinica antimaterije.

Iahub želi odabrati neki kontinuirani podniz uređaja u laboratoriji, odrediti način proizvodnje za svaki od njih (proizvodnja materije ili antimaterije) i na kraju napraviti fotografiju. Međutim, bit će uspješan samo ako su količine proizvedene materije i antimaterije u odabranom podnizu jednake (u suprotnom bi došlo do prekomjerne materije ili antimaterije na fotografiji).

\subsubsection{Ulaz}
\begin{itemize}
    \item Prvi red sadrži jedan cijeli broj $n$ ($1 \leq n \leq 1000$) -- broj uređaja
    \item Drugi red sadrži $n$ cijelih brojeva $a_1, a_2, ..., a_n$ ($1 \leq a_i \leq 1000$)
    \item Suma $a_1 + a_2 + ... + a_n$ će biti manja ili jednaka 10000
\end{itemize}

\subsubsection{Izlaz}
\begin{itemize}
    \item Ispisati jedan cijeli broj -- broj različitih načina na koje Iahub može napraviti fotografiju, po modulu $10^9 + 7$
\end{itemize}

\subsection{Pristup Rješenju}
Problem se rješava koristeći dinamičko programiranje. Ključna opservacija je da za svaki uređaj imamo dva izbora: proizvesti materiju ($+a_i$) ili antimateriju ($-a_i$). Cilj je pronaći sve kontinuirane podnizove gdje je suma odabranih vrijednosti jednaka nuli.

\subsection{Dinamičko Programiranje}
Definišemo stanje dp[i][s] gdje je:
\begin{itemize}
    \item $i$ - pozicija do koje smo obradili uređaje (0-bazirano)
    \item $s$ - trenutna razlika između količine materije i antimaterije
\end{itemize}

Koristimo tehniku "rolling array" za optimizaciju memorije, čuvajući samo dvije vrste stanja:
\begin{itemize}
    \item dp[current\_row][s] - stanja za trenutnu poziciju
    \item dp[next\_row][s] - stanja za sljedeću poziciju
\end{itemize}

\subsection{Tranzicije Stanja}
Za svaku poziciju $i$ i trenutnu sumu $s$, imamo sljedeće mogućnosti:
\begin{enumerate}
    \item Dodati materiju: dp[next\_row][s + a[i]] += dp[current\_row][s]
    \item Dodati antimateriju: dp[next\_row][s - a[i]] += dp[current\_row][s]
    \item Započeti novi podniz od trenutne pozicije: dp[next\_row][0] += 1
\end{enumerate}

\subsection{Analiza Kompleksnosti}
\begin{itemize}
    \item Vremenska Kompleksnost: $O(n \cdot MAXS)$
        \begin{itemize}
            \item Za svaku poziciju ($n$)
            \item Za svaku moguću sumu ($2 \cdot MAXS$)
            \item Gdje je MAXS = 10000
        \end{itemize}
    \item Prostorna Kompleksnost: $O(MAXS)$
        \begin{itemize}
            \item Koristimo samo dva reda DP tabele
            \item Svaki red ima veličinu $O(MAXS)$
        \end{itemize}
\end{itemize}

\subsection{Detalji Implementacije}
Implementacija koristi nekoliko ključnih optimizacija:
\begin{itemize}
    \item Korištenje "rolling array" tehnike za optimizaciju memorije
    \item Efikasno rukovanje negativnim indeksima pomoću pomaka u sredinu niza
    \item Pravilno rukovanje modularnom aritmetikom za velike brojeve
    \item Optimizovano čišćenje niza za sljedeću iteraciju
\end{itemize} 

\section{Problem 10: Minesweeper 1D (Codeforces 404 D- 1900 hard)}
\href{https://codeforces.com/contest/404/problem/D}{Link to problem}

\subsection{Postavka Problema}
Igra "Minesweeper 1D" se igra na liniji kvadrata, visine 1 kvadrat i širine $n$ kvadrata. Neki od kvadrata sadrže mine. Ako kvadrat ne sadrži minu, onda sadrži broj od 0 do 2 -- ukupan broj mina u susjednim kvadratima.

Na primjer, ispravno polje za igru izgleda ovako: 001*2***101*. Ćelije označene sa "*" sadrže mine. Primijetite da na ispravnom polju brojevi predstavljaju broj mina u susjednim ćelijama. Na primjer, polje 2* nije ispravno, jer ćelija sa vrijednošću 2 mora imati dvije susjedne ćelije sa minama.

Valera želi napraviti ispravno polje za igru "Minesweeper 1D". Već je nacrtao kvadratno polje širine $n$ ćelija, postavio nekoliko mina na polje i napisao brojeve u neke ćelije. Sada se pita na koliko načina može popuniti preostale ćelije minama i brojevima tako da na kraju dobije ispravno polje.

\subsubsection{Ulaz}
\begin{itemize}
    \item Prvi red sadrži niz karaktera bez razmaka $s_1s_2...s_n$ ($1 \leq n \leq 10^6$)
    \item Niz sadrži samo karaktere "*", "?" i cifre "0", "1" ili "2"
    \item Ako je karakter $s_i$ jednak "*", tada i-ta ćelija polja sadrži minu
    \item Ako je karakter $s_i$ jednak "?", tada Valera još nije odlučio šta staviti u i-tu ćeliju
    \item Karakter $s_i$ koji je jednak cifri predstavlja cifru napisanu u i-tom kvadratu
\end{itemize}

\subsubsection{Izlaz}
\begin{itemize}
    \item Ispisati jedan cijeli broj -- broj načina na koje Valera može popuniti prazne ćelije i dobiti ispravno polje
    \item Kako odgovor može biti velik, ispisati ga po modulu 1000000007 ($10^9 + 7$)
\end{itemize}

\subsection{Pristup Rješenju}
Problem se rješava koristeći dinamičko programiranje. Ključna opservacija je da za svaku ćeliju moramo pratiti broj mina u susjednim ćelijama kako bismo osigurali da brojevi u ćelijama budu ispravni.

\subsection{Dinamičko Programiranje}
Definišemo stanje dp[i][j][k] gdje je:
\begin{itemize}
    \item $i$ - pozicija do koje smo popunili polje
    \item $j$ - stanje prethodne ćelije (0-2 za broj, 3 za minu)
    \item $k$ - stanje trenutne ćelije (0-2 za broj, 3 za minu)
\end{itemize}

\subsection{Tranzicije Stanja}
Za svaku poziciju i stanje imamo sljedeće mogućnosti:
\begin{enumerate}
    \item Ako je trenutna ćelija "?", možemo staviti:
        \begin{itemize}
            \item Broj (0-2) ako je validan s obzirom na susjedne mine
            \item Minu (3) ako je validno s obzirom na susjedne brojeve
        \end{itemize}
    \item Ako je trenutna ćelija već popunjena, provjeravamo samo validnost
\end{enumerate}

\subsection{Analiza Kompleksnosti}
\begin{itemize}
    \item Vremenska Kompleksnost: $O(n \cdot 4 \cdot 4 \cdot 4)$
        \begin{itemize}
            \item Za svaku poziciju ($n$)
            \item Za svako stanje prethodne ćelije (4)
            \item Za svako stanje trenutne ćelije (4)
            \item Za svaku moguću vrijednost nove ćelije (4)
        \end{itemize}
    \item Prostorna Kompleksnost: $O(n \cdot 4 \cdot 4)$
        \begin{itemize}
            \item DP tabela: $O(n \cdot 4 \cdot 4)$
        \end{itemize}
\end{itemize}

\subsection{Detalji Implementacije}
Implementacija koristi nekoliko ključnih optimizacija:
\begin{itemize}
    \item Korištenje vrijednosti 3 za predstavljanje mine radi lakše obrade
    \item Efikasno računanje broja susjednih mina
    \item Provjera validnosti stanja prije ažuriranja DP tabele
    \item Pravilno rukovanje modularnom aritmetikom za velike brojeve
\end{itemize} 
\end{document} 